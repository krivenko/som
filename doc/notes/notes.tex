\documentclass[]{article}
\usepackage{amsmath}
\usepackage{graphicx}
\usepackage[bookmarks]{hyperref}
\usepackage[a4paper, total={7in, 8in}]{geometry}

\newcommand{\intt}{\ensuremath{\int\limits_0^\beta}\ d\tau}
\DeclareMathOperator{\atan}{atan}
\DeclareMathOperator{\sign}{sign}

% Title Page
\title{Stochastic Optimization Method for Analytic Continuation: implementation notes}
\author{Igor Krivenko}

\begin{document}
\maketitle

This document is a diverse collection of notes, explaining various details of the present SOM implementation.
It is supposed to serve as an addition to the lecture \cite{som_lecture}.

\section{Supported integral kernels}
\label{kernels}
\subsection{Fermionic thermal Green's function in imaginary time}
\label{fermiongf_imtime}

\begin{equation}
	G(\tau) = -\int\limits_{-\infty}^\infty
	d\epsilon \frac{e^{-\tau\epsilon}}{1+e^{-\beta\epsilon}} A(\epsilon).
\end{equation}

Integral kernel applied to a rectangle,
\begin{equation}
	\hat K(\tau;\omega)*R(c,w,h) =
	h(\Lambda(\tau;c+w/2) - \Lambda(\tau;c-w/2)),
\end{equation}
\begin{equation}
	\Lambda(\tau;\Omega) = -\frac{1}{\beta}
	\int\limits_0^{\beta\Omega}\frac{e^{-(\tau/\beta)x}}{1+e^{-x}}dx.
\end{equation}
The integral on the right hand side is analytically doable only for
a few values of $\alpha=\tau/\beta$, namely $\alpha=0,1/2,1$.
For all other $\alpha\in(0;1)$, it has to be done numerically and approximated using a cubic spline interpolation. The spline interpolation
would be easy to construct if the integrand were localized near $x=0$ for
all $\alpha$. However, this is not the case. The integrand develops
a long ``tail'' on the positive/negative half-axis as $\alpha$ approaches
0/1 respectively. The length of this tail scales as $\alpha^{-1}$ (or $(1-\alpha)^{-1}$), which would require an excessively large number of
spline knots in order to construct a reliable approximation.

The issue is solved by subtracting a tail contribution $t_\alpha(x)$ from the integrand, such that the difference is well localized, and an integral of $t_\alpha(x)$ is trivial.
\begin{equation}
	\Lambda(\tau;\Omega) = -\frac{1}{\beta} \left[
	\theta(-\Omega)S^-_\alpha(\beta\Omega) +
	\theta(\Omega)S^+_\alpha(\beta\Omega) +
	T_\alpha(\beta\Omega)
	\right],
\end{equation}
\begin{equation}
	S^-_\alpha(z) = -\int\limits_z^0
	\left[\frac{e^{-\alpha x}}{1+e^{-x}} - t_\alpha(x)
	\right] dx,\quad
	S^+_\alpha(z) = \int\limits_0^z
	\left[\frac{e^{-\alpha x}}{1+e^{-x}} - t_\alpha(x)
	\right] dx,
\end{equation}
\begin{equation}
	T_\alpha(z) = \int\limits_0^z t_\alpha(x) dx.
\end{equation}

The integrals $S^-_\alpha(z)$ and $S^+_\alpha(z)$ are precomputed on
a mesh with a fixed number of points, and provide input data for
spline construction. Limits of the meshes are $\alpha$-independent,
\begin{equation}
	S^-_\alpha(z): z\in[-x_0,0], \quad
	S^+_\alpha(z): z\in[0;x_0], \quad
	x_0 = -2\log(\mathtt{tolerance}).
\end{equation}
$x_0$ is chosen for the least localized case of $\alpha=1/2$.
We need two different splines (functions $S^\pm_\alpha(z)$) for positive and negative $z$ because the integrand is discontinuous (except for $\alpha=1/2$).

The tail contributions $t_\alpha(x)$ and all related quantities we use are
summarized in a table below.

\newcommand{\auxsum}[1]{\ensuremath{\sum_{n=0}^\infty(-1)^n\frac{e^{#1z}}{#1}}}
\begin{center}
\begin{tabular}{|c|c|c|c|c|}
\hline
$\alpha$ & $t_\alpha(x)$ & $T_\alpha(z)$ & $S_\alpha^-(z)$ & $S_\alpha^+(z)$\\
\hline
$0$ & $\theta(x)$ & $\theta(z)z$ &
$\log(1+e^{z})-\log(2)$ & $\log(1+e^{-z}) - \log(2)$ \\
\hline
$(0;1/2)$ & $\theta(x)e^{-\alpha x}$ & $\theta(z)\frac{e^{-\alpha z}-1}{-\alpha}$ &
$\auxsum{(n+1-\alpha)}-\Psi(1-\alpha)$ & $-\auxsum{-(n+1+\alpha)}-\Psi(1+\alpha)$ \\
\hline
$1/2$ & $0$ & $0$ & $2\atan(e^{z/2})-\pi/2$ & $-2\atan(e^{-z/2})+\pi/2$ \\
\hline
$(1/2;1)$ & $\theta(-x)e^{(1-\alpha)x}$ & $\theta(-z)\frac{e^{(1-\alpha)z}-1}{1-\alpha}$ & $-\auxsum{(n+2-\alpha)}+\Psi(2-\alpha)$ & $\auxsum{-(n+\alpha)}+\Psi(\alpha)$ \\
\hline
$1$ & $\theta(-x)$ & $\theta(-z)z$ &
$-\log(1+e^z) + \log(2)$ & $-\log(1+e^{-z}) + \log(2)$ \\
\hline
\end{tabular}
\end{center}

All series found in the table are rapidly convergent, $\Psi(z) = \frac{1}{2}[\psi(\frac{1+z}{2}) - \psi(\frac{z}{2})]$, $\psi(z)$ is the digamma function.

\subsection{Fermionic thermal Green's function at imaginary frequencies}
\label{fermiongf_imfreq}
\begin{equation}
	G(i\omega_n) = \int\limits_{-\infty}^\infty
	d\epsilon \frac{1}{i\omega_n-\epsilon} A(\epsilon).
\end{equation}

Integral kernel applied to a rectangle,
\begin{equation}
	\hat K(i\omega_n;\omega)*R(c,w,h) =
	h \log\left(\frac{i\omega_n - c + w/2}{i\omega_n - c - w/2}\right).
\end{equation}

\subsection{Fermionic thermal Green's function in Legendre polynomial basis}
\label{fermiongf_legendre}
\begin{equation}
	G(\ell) = -\int\limits_{-\infty}^\infty
	d\epsilon \frac{\beta\sqrt{2\ell+1}(-\mathrm{sgn}(\epsilon))^\ell i_{\ell}(\beta|\epsilon|/2)}
	{2\cosh(\beta\epsilon/2)} A(\epsilon)
\end{equation}
$i_\ell(x)$ is the modified spherical Bessel function of the first kind.

Integral kernel applied to a rectangle,
\begin{equation}
	\hat K(\ell;\omega)*R(c,w,h) = h (\Lambda(\ell;c+w/2) - \Lambda(\ell;c-w/2))
\end{equation}
\begin{equation}
	\Lambda(\ell;\Omega) = (-\mathrm{sgn}(\Omega))^{\ell+1}\sqrt{2\ell+1}
	\int_0^{|\Omega|\beta/2} \frac{i_\ell(x)}{\cosh(x)} dx
\end{equation}

The integral $\int_0^z \frac{i_\ell(x)}{\cosh(x)} dx$ is rather inconvenient for fast numerical evaluation. It cannot be expressed in terms of elementary functions or of a quickly converging series. Moreover, it grows logarithmically for $z\to\infty$, which makes spline interpolation infeasible.

These difficulties urge us to use a combined evaluation scheme. The scheme is based on the following expression for the integrand:
\begin{equation}\label{il_cosh_series}
	\frac{i_\ell(x)}{\cosh(x)} =
	\frac{e^x}{e^x+e^{-x}}\sum_{k=0}^\ell(-1)^k
		\frac{a_k(\ell+1/2)}{x^{k+1}} +
	\frac{e^{-x}}{e^x+e^{-x}}\sum_{k=0}^\ell(-1)^{\ell+1}
		\frac{a_k(\ell+1/2)}{x^{k+1}},
\end{equation}
where $a_k(\ell+1/2)$ are coefficients of the Bessel polynomials,
\begin{equation}
	a_k(\ell+1/2) = \frac{(l+k)!}{2^k k!(l-k)!} =
	\prod_{i=1}^k \frac{1}{2i}[\ell-k+2i-1][\ell-k+2i].
\end{equation}

For large $x$ (high-frequency region) the integrand can be approximated as
\begin{equation}\label{il_cosh_series_high}
	\frac{i_\ell(x)}{\cosh(x)} \approx
		\sum_{k=0}^\ell(-1)^k \frac{a_k(\ell+1/2)}{x^{k+1}}.
\end{equation}

For each $\ell$ the boundary value $x_0$ between the low- and high- frequency regions is found by a direct evaluation of (\ref{il_cosh_series}) and (\ref{il_cosh_series_high}). $x_0\approx15$ for $\ell=0$ (discrepancy on the level of $10^{-13}$) and grows as $(\mathrm{const}+\ell)^2$ for higher $\ell$.

In the low-energy region we do the integral $F^<(z) \equiv \int_0^z \frac{i_\ell(x)}{\cosh(x)} dx$ on a fixed $z$-mesh, $z\in[0;x_0]$, using the adaptive Simpson's method. Results of the integrations are used to construct a cubic spline interpolation of $F^<(z)$.

In the high-energy region we use the asymptotic form (\ref{il_cosh_series_high}) to do the integral,
\begin{equation}
	F^>(z)|_{z>x_0} = F^<(x_0) +
		\int_{x_0}^z dx \sum_{k=0}^\ell(-1)^k \frac{a_k(\ell+1/2)}{x^{k+1}},
\end{equation}
\begin{equation}
	\int_{x_0}^z dx \sum_{k=0}^\ell(-1)^k \frac{a_k(\ell+1/2)}{x^{k+1}} =
	\left.\left\{
		\log(x) +
		\sum_{k=1}^\ell (-1)^{k+1}\frac{a_k(\ell+1/2)}{x^k k}
	\right\}\right|_{x_0}^z.
\end{equation}

\subsection{Correlator of boson-like operators in imaginary time}
\label{bosoncorr_imtime}
\begin{equation}
	\chi(\tau) = \int\limits_{-\infty}^\infty \frac{d\epsilon}{\pi}
	\frac{\epsilon e^{-\tau\epsilon}}{1-e^{-\beta\epsilon}} A(\epsilon).
\end{equation}

Integral kernel applied to a rectangle,
\begin{equation}
\hat K(\tau;\omega)*R(c,w,h) =
h(\Lambda(\tau;c+w/2) - \Lambda(\tau;c-w/2)),
\end{equation}
\begin{equation}
\Lambda(\tau;\Omega) = \frac{1}{\pi\beta^2}
\int\limits_0^{\beta\Omega}\frac{xe^{-(\tau/\beta)x}}{1-e^{-x}}dx.
\end{equation}

Here we employ a tail subtraction technique analogous to that of paragraph
\ref{fermiongf_imtime}. All notations remain similar.

\begin{equation}
	\Lambda(\tau;\Omega) = \frac{1}{\pi\beta^2} \left[
	\theta(-\Omega)S^-_\alpha(\beta\Omega) +
	\theta(\Omega)S^+_\alpha(\beta\Omega) +
	T_\alpha(\beta\Omega)
	\right],
\end{equation}
\begin{equation}
	S^-_\alpha(z) = -\int\limits_z^0
	\left[\frac{xe^{-\alpha x}}{1-e^{-x}} - t_\alpha(x)
	\right] dx,\quad
	S^+_\alpha(z) = \int\limits_0^z
	\left[\frac{xe^{-\alpha x}}{1-e^{-x}} - t_\alpha(x)
	\right] dx,
\end{equation}
\begin{equation}
	T_\alpha(z) = \int\limits_0^z t_\alpha(x) dx.
\end{equation}

\newcommand{\dilog}{\ensuremath{\mathrm{Li}_2}}
\begin{center}
	\footnotesize
	\begin{tabular}{|c|c|c|c|c|}
		\hline
		$\alpha$ & $t_\alpha(x)$ & $T_\alpha(z)$ & $S_\alpha^-(z)$ & $S_\alpha^+(z)$\\
		\hline
		$0$ & $\theta(x)x$ & $\theta(z)z^2/2$ &
		$-\pi^2/6+z\log(1-e^z)+\dilog(e^z)$ & $\pi^2/6+z\log(1-e^{-z})-\dilog(e^{-z})$ \\
		\hline
		$(0;1/2)$ & $\theta(x)xe^{-\alpha x}$ &
		$\theta(z)\frac{1-e^{-\alpha z}(1+z\alpha)}{\alpha^2}$ &
		$\begin{aligned}\sum_{n=0}^\infty\frac{e^{(n+1-\alpha)z}[1-z(n+1-\alpha)]}{(n+1-\alpha)^2}
		-\\-\psi^{(1)}(1-\alpha)\end{aligned}$ & $\begin{aligned}-\sum_{n=0}^\infty\frac{e^{-(n+1+\alpha)z}[1+z(n+1+\alpha)]}{(n+1+\alpha)^2}
		+\\+\psi^{(1)}(1+\alpha)\end{aligned}$ \\
		\hline
		$1/2$ & $0$ & $0$ & $\begin{aligned}-\pi^2/2+z\log(\tanh(-z/4))-\\-\dilog(e^z)+4\dilog(e^{z/2})\end{aligned}$ &
		$\begin{aligned}\pi^2/2+z\log(\tanh(z/4))+\\+\dilog(e^{-z})-4\dilog(e^{-z/2})\end{aligned}$ \\
		\hline
		$(1/2;1)$ & $\theta(-x)(-x)e^{(1-\alpha)x}$ &
		$\theta(-z)\frac{e^{(1-\alpha)z}(1-z(1-\alpha))-1}{(1-\alpha)^2}$ &
		$\begin{aligned}\sum_{n=0}^\infty\frac{e^{(n+2-\alpha)z}[1-z(n+2-\alpha)]}{(n+2-\alpha)^2}
		-\\-\psi^{(1)}(2-\alpha)\end{aligned}$ & $\begin{aligned}-\sum_{n=0}^\infty\frac{e^{-(n+\alpha)z}[1+z(n+\alpha)]}{(n+\alpha)^2}
		+\\+\psi^{(1)}(\alpha)\end{aligned}$ \\
		\hline
		$1$ & $\theta(-x)(-x)$ & $\theta(-z)(-z^2/2)$ &
		$-\pi^2/6+z\log(1-e^z)+\dilog(e^z)$ & $\pi^2/6+z\log(1-e^{-z})-\dilog(e^{-z})$ \\
		\hline
	\end{tabular}
\end{center}

All series found in the table are rapidly convergent, $\dilog(z)$ is the dilogarithm (Spence's function),
$\psi^{(1)}(z) = \frac{d^2}{dz^2}\log\Gamma(z)$ is the trigamma function.

Factor $x$ in the kernel tend to strengthen its delocalization, so
it makes sense to choose $x_0$ (boundary of the spline interpolation region)
a bit larger, for instance, $x_0=-2.3\log(\mathtt{tol})$.

\subsection{Correlator of boson-like operators at imaginary frequencies}
\label{bosoncorr_imfreq}
\begin{equation}
\chi(i\Omega_n) = \int\limits_{-\infty}^\infty
d\epsilon \frac{1}{\pi}\frac{-\epsilon}{i\Omega_n-\epsilon} A(\epsilon).
\end{equation}

Integral kernel applied to a rectangle,
\begin{equation}
\hat K(i\Omega_n;\omega)*R(c,w,h) =
\frac{hw}{\pi} + \frac{hi\Omega_n}{\pi}\log\left(\frac{i\Omega_n - c - w/2}{i\Omega_n - c + w/2}\right).
\end{equation}

\subsection{Correlator of boson-like operators in Legendre polynomial basis}
\label{bosoncorr_legendre}
\begin{equation}
	\chi(\ell) = \int\limits_{-\infty}^\infty
	\frac{d\epsilon}{\pi}
	\frac{\beta\epsilon\sqrt{2\ell+1}(-\mathrm{sgn}(\epsilon))^\ell i_{\ell}(\beta|\epsilon|/2)}
	{2\sinh(\beta\epsilon/2)} A(\epsilon).
\end{equation}

Integral kernel applied to a rectangle,
\begin{equation}
	\hat K(\ell;\omega)*R(c,w,h) = h (\Lambda(\ell;c+w/2) - \Lambda(\ell;c-w/2))
\end{equation}
\begin{equation}
	\Lambda(\ell;\Omega) = -(-\mathrm{sgn}(\Omega))^{\ell+1}
	\frac{2\sqrt{2\ell+1}}{\pi\beta}
	\int_0^{|\Omega|\beta/2} \frac{i_\ell(x) x}{\sinh(x)} dx
\end{equation}

Once again, we apply the combined evaluation scheme described in
\ref{fermiongf_legendre}. An explicit expression for the integrand reads
\begin{equation}\label{il_x_sinh_series}
	\frac{i_\ell(x) x}{\sinh(x)} =
	\frac{e^x}{e^x-e^{-x}}\sum_{k=0}^\ell(-1)^k
	\frac{a_k(\ell+1/2)}{x^k} +
	\frac{e^{-x}}{e^x-e^{-x}}\sum_{k=0}^\ell(-1)^{\ell+1}
	\frac{a_k(\ell+1/2)}{x^k}.
\end{equation}
Its high-frequency asymptotics is
\begin{equation}
	\frac{i_\ell(x) x}{\sinh(x)} \approx
		\sum_{k=0}^\ell(-1)^k \frac{a_k(\ell+1/2)}{x^k}.
\end{equation}

Integral $F^<(z) \equiv \int_0^z \frac{i_\ell(x)x}{\sinh(x)} dx$ is done numerically,
and replaced by a cubic spline. For $z>x_0$ we have to add a tail contribution,
\begin{equation}
F^>(z)|_{z>x_0} = F^<(x_0) +
\left.\left\{
x - \frac{\ell(\ell+1)}{2}\log(x) +
\sum_{k=1}^{\ell-1} (-1)^k\frac{a_{k+1}(\ell+1/2)}{x^k k}
\right\}\right|_{x_0}^z.
\end{equation}

\subsection{Autocorrelator of a boson-like operator in imaginary time}
\label{bosonautocorr_imtime}
\begin{equation}
\chi(\tau) = \int\limits_{-\infty}^\infty \frac{d\epsilon}{\pi}
\frac{\epsilon (e^{-\tau\epsilon}+e^{-(\beta-\tau)\epsilon})}
{1-e^{-\beta\epsilon}} A(\epsilon).
\end{equation}

Integral kernel applied to a rectangle,
\begin{equation}
\hat K(\tau;\omega)*R(c,w,h) =
h(\Lambda(\tau;c+w/2) - \Lambda(\tau;c-w/2)),
\end{equation}
\begin{equation}
\Lambda(\tau;\Omega) = \frac{1}{\pi\beta^2}
\int\limits_0^{\beta\Omega}\frac{x(e^{-(\tau/\beta)x}+e^{-(1-\tau/\beta)x})}
{1-e^{-x}}dx.
\end{equation}
This integral is obtained from the one of paragraph \ref{bosoncorr_imtime}
as a result of $\tau$-symmetrization (sum of $\Lambda$ for $\alpha$ and $1-\alpha$).
Thanks to this fact, it is enough to consider only 3 $\alpha$-cases,
\begin{itemize}
	\item $\alpha=0,1$;
	\item $\alpha=1/2$;
	\item $\alpha\in(0;1/2)\cup(1/2;1)$.
\end{itemize}
Furthermore, the upper integration limit can now be only positive, which
eliminates the need in two cubic splines.

\begin{equation}
\Lambda(\tau;\Omega) = \frac{1}{\pi\beta^2} \left[
S_\alpha(\beta\Omega) + T_\alpha(\beta\Omega) \right],
\end{equation}
\begin{equation}
S_\alpha(z) = \int\limits_0^z
\left[\frac{x(e^{-\alpha x}+e^{-(1-\alpha)x})}{1-e^{-x}} - t_\alpha(x)
\right] dx, \quad
T_\alpha(z) = \int\limits_0^z t_\alpha(x) dx.
\end{equation}

\begin{center}
	\footnotesize
	\begin{tabular}{|c|c|c|c|}
		\hline
		$\alpha$ & $t_\alpha(x)$ & $T_\alpha(z)$ & $S_\alpha(z)$ \\
		\hline
		$0,1$ & $x(1+e^{-x})$ & $1+z^2/2-e^{-z}(1+z)$ &
		$-1+\pi^2/3+2z\log(1-e^{-z})-2\dilog(e^{-z})+e^{-z}(1+z)$ \\
		\hline
		$(0;1/2)\cup(1/2;1)$ & $x(e^{-\alpha x}+e^{-(1-\alpha)x})$ &
		$\begin{array}{c}\frac{1-e^{-\alpha z}(1+z\alpha)}{\alpha^2}+\\+
		\frac{1-e^{-(1-\alpha) z}(1+z(1-\alpha))}{(1-\alpha)^2}\end{array}$ &$\begin{aligned}-\sum_{n=0}^\infty\frac{e^{-(n+1+\alpha)z}[1+z(n+1+\alpha)]}
		{(n+1+\alpha)^2} +\psi^{(1)}(1+\alpha) -\\
		-\sum_{n=0}^\infty\frac{e^{-(n+2-\alpha)z}[1+z(n+2-\alpha)]}
		{(n+2-\alpha)^2} +\psi^{(1)}(2-\alpha)\end{aligned}$ \\
		\hline
		$1/2$ & $2xe^{-x/2}$ & $4(2-e^{-z/2}(2+z))$ &
		$\begin{aligned}-8+\pi^2+4e^{-z/2}(2+z)+2z\log\tanh(z/4)-\\
		-8\dilog(e^{-z/2})+2\dilog(e^{-z})\end{aligned}$ \\
		\hline
	\end{tabular}
\end{center}

All series found in the table are rapidly convergent, $\dilog(z)$ is the dilogarithm (Spence's function),
$\psi^{(1)}(z) = \frac{d^2}{dz^2}\log\Gamma(z)$ is the trigamma function.

\subsection{Autocorrelator of a boson-like operator at imaginary frequencies}
\label{bosonautocorr_imfreq}
\begin{equation}
\chi(i\Omega_n) = \int\limits_0^\infty
d\epsilon \frac{1}{\pi}\frac{2\epsilon^2}{\Omega_n^2+\epsilon^2} A(\epsilon).
\end{equation}

Integral kernel applied to a rectangle,
\begin{equation}
\hat K(i\Omega_n;\omega)*R(c,w,h) =
\frac{2hw}{\pi} + \frac{2h\Omega_n}{\pi}\left(
\atan\left(\frac{c-w/2}{\Omega_n}\right) - \atan\left(\frac{c+w/2}{\Omega_n}\right)
\right).
\end{equation}

\subsection{Autocorrelator of a boson-like operator in Legendre polynomial basis}
\label{bosonautocorr_legendre}
\begin{equation}
	\chi(\ell) = \left\{
		\begin{array}{ll}
		\int\limits_0^\infty
		\frac{d\epsilon}{\pi}
		\frac{\beta\epsilon\sqrt{2\ell+1} i_{\ell}(\beta\epsilon/2)}
		{\sinh(\beta\epsilon/2)} A(\epsilon), &\ell\ \mathrm{ even},\\
		0, &\ell\ \mathrm{odd}.
	\end{array}\right.
\end{equation}

Integral kernel applied to a rectangle,
\begin{equation}
	\hat K(\ell;\omega)*R(c,w,h) = \left\{
	\begin{array}{ll}
	h(\Lambda(\ell;c+w/2)-\Lambda(\ell;c-w/2)), &\ell\ \mathrm{ even},\\
	0, &\ell\ \mathrm{odd}.
	\end{array}\right.
\end{equation}
\begin{equation}
	\Lambda(\ell;\Omega) =
	\frac{4\sqrt{2\ell+1}}{\pi\beta}
	\int_0^{\Omega\beta/2} \frac{i_\ell(x) x}{\sinh(x)} dx.
\end{equation}

See paragraph \ref{bosoncorr_legendre} for a description of how the integral on
the right hand side is evaluated.

\subsection{Zero temperature correlator in imaginary time}
\label{zerotemp_imtime}
\begin{equation}
	G(\tau) = -\int\limits_0^\infty
	d\epsilon\ e^{-\tau\epsilon} A(\epsilon), \quad
	\tau\in[0;\tau_{max}]
\end{equation}

Integral kernel applied to a rectangle,
\begin{equation}
	\hat K(\tau;\omega)*R(c,w,h) =  \left\{
	\begin{array}{ll}
		-hw, &\tau=0,\\
		\frac{h}{\tau}(e^{-\tau(c+w/2)}-e^{-\tau(c-w/2)}) ,&\mathrm{otherwise}.
	\end{array}\right.
\end{equation}

\subsection{Zero temperature correlator at imaginary frequencies}
\label{zerotemp_imfreq}
\begin{equation}
	G(i\omega_n) = \int\limits_0^\infty
	d\epsilon \frac{1}{i\omega_n-\epsilon} A(\epsilon).
\end{equation}

Imaginary frequency values $\omega_n$ are arbitrary here, with a restriction that
$\omega_0 = 0$ is not included in the case of fermionic correlators.

Integral kernel applied to a rectangle,
\begin{equation}
\hat K(i\omega_n;\omega)*R(c,w,h) =
h \log\left(\frac{i\omega_n - c + w/2}{i\omega_n - c - w/2}\right).
\end{equation}

\subsection{Zero temperature correlator in Legendre polynomial basis}
\label{zerotemp_legendre}
\begin{equation}
	G(\ell) =
	\int\limits_0^\infty d\epsilon
	\tau_{max}(-1)^{\ell+1}\sqrt{2\ell+1}
	i_{\ell}\left(\frac{\epsilon\tau_{max}}{2}\right)
	\exp\left(-\frac{\epsilon\tau_{max}}{2}\right) A(\epsilon).
\end{equation}

Integral kernel applied to a rectangle,
\begin{equation}
	\hat K(\ell;\omega)*R(c,w,h) = h (\Lambda(\ell;c+w/2) - \Lambda(\ell;c-w/2)),
\end{equation}
\begin{equation}
	\Lambda(\ell;\Omega) = (-1)^{\ell+1}\sqrt{2\ell+1}
	\int_0^{\Omega\tau_{max}/2} 2e^{-x} i_\ell(x) dx.
\end{equation}

Once again, we apply the combined evaluation scheme described in
\ref{fermiongf_legendre}. An explicit expression for the integrand reads
\begin{equation}\label{il_x_exp_series}
	2e^{-x} i_\ell(x) =
	\sum_{k=0}^\ell(-1)^k
	\frac{a_k(\ell+1/2)}{x^k} +
	e^{-2x}\sum_{k=0}^\ell(-1)^{\ell+1}
	\frac{a_k(\ell+1/2)}{x^k}.
\end{equation}
Its high-frequency asymptotics is
\begin{equation}
	2e^{-x} i_\ell(x) \approx
	\sum_{k=0}^\ell(-1)^k \frac{a_k(\ell+1/2)}{x^k}.
\end{equation}

Integral $F^<(z) \equiv \int_0^z 2 e^{-x} i_\ell(x) dx$ is done numerically,
and replaced by a cubic spline. For $z>x_0$ we have to add a tail contribution,
\begin{equation}
F^>(z)|_{z>x_0} = F^<(x_0) + \left.\left\{
\log(x) + \sum_{k=1}^\ell (-1)^{k+1}\frac{a_k(\ell+1/2)}{x^k k}
\right\}\right|_{x_0}^z.
\end{equation}

\section{Fit quality criterion for complex functions}
\label{fit_quality}

In section 4.1 Mishchenko introduces a special quantity $\kappa$, which characterizes the fit quality of a given particular solution
$A(\omega)$ (eq. 55).
\begin{equation}
	\kappa = \frac{1}{M-1}\sum_{m=2}^M\theta(-\Delta(m)\Delta(m-1)),
\end{equation}
where $\Delta(m) = (\tilde G(m) - G(m))/\mathcal{S}(m)$ is the deviation function. The proposed expression for $\kappa$ makes  apparently no sense for complex $G(m)$ and $\Delta(m)$, e.g. when $G$ is given as a function of Matsubara frequencies.

Our generalization consists in a replacement
\begin{multline}
	\theta(-\Delta(m)\Delta(m-1)) \mapsto
	\frac{1}{2}\left[1 - \frac{\Re[\Delta(m)\Delta^*(m-1)]}{|\Delta(m)\Delta^*(m-1)|} \right]=\\=
	\frac{1}{2}\left[ 1 - \cos[\arg(\Delta(m)) - \arg(\Delta(m-1))] \right].
\end{multline}
According to the modified definition, two adjacent values of $\Delta$ are considered anti-correlated, if the complex phase shift between them exceeds $\pm\pi/2$. In the case of real-valued quantities the phase shift is always either 0, or $\pi$.

\section{Probability density function for the parameter change}
\label{prob_function}

Section 3.4 of \cite{som_lecture} contains a dubious expression for the probability
density function $\mathcal{P}\sim (\max(|\delta\xi_\mathrm{min}|, |\delta\xi_\mathrm{max}|)/|\delta\xi|)^\gamma$, where $\gamma\gg1$. It is
obviously divergent at $\delta\xi\to0$ and cannot be properly normalized, if
ends of $[\delta\xi_\mathrm{min};\delta\xi_\mathrm{max}]$ segment have different signs.

We use a different density function, which is finite everywhere on $\delta\xi\in[\delta\xi_\mathrm{min};\delta\xi_\mathrm{max}]$, and is qualitatively similar to that of Mishchenko.
\begin{equation}
	\mathcal{P}(\delta\xi) = \mathcal{N}
	\exp\left(-\gamma \frac{|\delta\xi|}{L}\right), \quad
	L \equiv \max(|\delta\xi_\mathrm{min}|, |\delta\xi_\mathrm{max}|),
\end{equation}
\begin{equation}
	\mathcal{N} = \frac{\gamma}{L}\left[
		\sign(\delta\xi_\mathrm{min})(e^{-\gamma|\delta\xi_\mathrm{min}|/L} - 1) +
		\sign(\delta\xi_\mathrm{max})(1 - e^{-\gamma|\delta\xi_\mathrm{max}|/L})
	\right]^{-1}.
\end{equation}

\section{Hilbert(-like) transforms of a rectangular spectral function}
\label{hilbert_transform}

Recovery of observables as functions of the real frequency from a computed spectral function amounts to evaluating Hilbert(-like) transforms of rectangles $R(c,w,h)$. The transforms in question are
\begin{itemize}
    \item
    \begin{equation}
        F_1(\epsilon) =
        - \int_{-\infty}^\infty d\epsilon'
            \frac{R(c,w,h;\epsilon')}{\epsilon' - \epsilon - i0} =
        -h \int_{c-w/2}^{c+w/2} d\epsilon' \frac{1}{\epsilon' - \epsilon - i0} =
        -h \log\left(\frac{c + w / 2 - \epsilon}{c - w / 2 - \epsilon}\right)
    \end{equation}
    for fermionic thermal Green's functions and zero temperature correlators;

    \item
    \begin{equation}
        F_2(\epsilon) =
        - \int_{-\infty}^\infty d\epsilon'
        \frac{\epsilon' R(c,w,h;\epsilon')}{\epsilon' - \epsilon - i0} =
        -h \int_{c-w/2}^{c+w/2} d\epsilon'
            \frac{\epsilon'}{\epsilon' - \epsilon - i0} =
        -h \left[ w + \epsilon \log\left(\frac{c + w / 2 - \epsilon}{c - w / 2 - \epsilon}\right)\right]
    \end{equation}
    for (auto)correlators of boson-like operators.
\end{itemize}

When binning is used to perform projection of an observable onto a real frequency mesh, functions $F_1(\epsilon)$ and $F_2(\epsilon)$ are replaced by their averages over a finite segment $[\epsilon_a, \epsilon_b]$.
\begin{multline}
    \frac{1}{\epsilon_b - \epsilon_a} \int_{\epsilon_a}^{\epsilon_b}
        F_1(\epsilon) d\epsilon =\\=
        \frac{h}{\epsilon_b - \epsilon_a} \left[
            (\epsilon_- - \epsilon_a) \log(\epsilon_- - \epsilon_a) +
            (\epsilon_+ - \epsilon_b) \log(\epsilon_+ - \epsilon_b) -
            (\epsilon_+ - \epsilon_a) \log(\epsilon_+ - \epsilon_a) -
            (\epsilon_- - \epsilon_b) \log(\epsilon_- - \epsilon_b)
        \right],
\end{multline}
\begin{multline}
    \frac{1}{\epsilon_b - \epsilon_a} \int_{\epsilon_a}^{\epsilon_b}
    F_2(\epsilon) d\epsilon = -\frac{hw}{2} -\\-
    \frac{h}{2(\epsilon_b - \epsilon_a)} \left[
        (\epsilon_-^2 - \epsilon_b^2) \log(\epsilon_- - \epsilon_b) +
        (\epsilon_+^2 - \epsilon_a^2) \log(\epsilon_+ - \epsilon_a) -
        (\epsilon_+^2 - \epsilon_b^2) \log(\epsilon_+ - \epsilon_b) -
        (\epsilon_-^2 - \epsilon_a^2) \log(\epsilon_- - \epsilon_a)
    \right],
\end{multline}
where $\epsilon_\pm = c \pm w/2$ are edges of the rectangle.

\begin{thebibliography}{9}
\bibitem{som_lecture}

Eva Pavarini, Erik Koch, Frithjof Anders, and Mark Jarrell (eds.)
{\it Correlated Electrons: From Models to Materials
Modeling and Simulation}, Vol. 2, Chapter 14.
Verlag des Forschungszentrum Jülich, 2012,
ISBN 978-3-89336-796-2
\url{https://www.cond-mat.de/events/correl12/manuscripts/mishchenko.pdf}

\end{thebibliography}

\end{document}
